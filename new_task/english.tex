\usepackage{xcolor}
\usepackage{afterpage}
\usepackage{pifont,mdframed}
\usepackage[bottom]{footmisc}

\makeatletter
\gdef\this@inputfilename{input}
\gdef\this@outputfilename{output}
\makeatother

    This is a nice task.

    \begin{figure}[H]
        \centering
        \includegraphics[width=0.9\linewidth]{nome.jpg}
        \caption{This is a nice caption.}
    \end{figure}

    This is a nice task.

\begin{warning}
Among the attachments of this task you may find a template file \texttt{nome.*} with a sample incomplete implementation.
\end{warning}

\InputFile
The first line contains the only integer $N$. The second line contains $N$ integers $V_i$.

\OutputFile
You need to write a single line with an integer: the unique integer that solves this task.

\begin{danger}
The \emph{modulo} operation $(a \bmod m)$ can be written in C/C++/Python as \verb|(a % m)| and in Pascal as \verb|(a mod m)|. To avoid the \emph{integer overflow} error, remember to reduce all partial results through the modulus, and not just the final result! \\
\emph{Notice that if $x < 10^9 + 7$, then $2 x$ fits into a} C/C++ \verb|int| \emph{and} Pascal \verb|longint|.
\end{danger}

% Assunzioni
\Constraints
\begin{itemize}[nolistsep, itemsep=2mm]
	\item $1 \le N \le 10\,000$.
	\item $1 \le V_i \le 100\,000$ for each $i=0\ldots N-1$.
\end{itemize}

\Scoring
Your program will be tested against several test cases grouped in subtasks.
In order to obtain the score of a subtask, your program needs to correctly solve all of its test cases.

\OISubtask{0}{1}{Examples.}

\OISubtask{30}{1}{$N \le 10$.}

\OISubtask{50}{1}{$N \le 100$.}

\OISubtask{20}{1}{No additional limitations.}

% Esempi


\Examples
\begin{example}
\exmpfile{nome.input0.txt}{nome.output0.txt}%
\exmpfile{nome.input1.txt}{nome.output1.txt}%
\end{example}


\Explanation
In the \textbf{first sample case}.

In the \textbf{second sample case}.
